\documentclass{ctexart}
\usepackage{amsmath}
\begin{document}
\section*{虚拟存储器综合应用题08}
\subsection*{若数据区大小为8 KB,}
\noindent
每块$\text{int}$数据数为$32\ \text{B}\div (\text{sizeof}\ (\text{int})\ \text{B})=8$;\\
数组a每行块数为$64\div 8=8$;\\
Cache行数为$8\ \text{KB}\div 32\ \text{B}=2^8$;\\
Cache组数为$2^8\div 4=64$;\\
数组a块内地址为$0\:0000B$,在块边界处;\\
行优先访问,则按块依次访问,命中率为$(8-1)\div 8=87.5\%$;\\
若讲i,j次序调换,则为列优先访问,前$8$行每块与Cache每组一一对应,后两个连续8行访问时调入新主存块,但Cache每组行数4大于3,故不发生替换,命中率仍为$(8-1)\div 8=87.5\%$。
\subsection*{若数据区大小为4 KB,}
\noindent
每块$\text{int}$数据数为$32\text{B}\div (\text{sizeof}\ (\text{int})\ \text{B})=8$;\\
数组a每行块数为$64\div 8=8$;\\
Cache行数为$4\ \text{KB}\div 32\ \text{B}=2^7$;\\
Cache组数为$2^7\div 4=32$;\\
数组a块内地址为$0\:0000B$,在块边界处;\\
行优先访问,则按块依次访问,命中率为$(8-1)\div 8=87.5\%$;\\
若讲i,j次序调换,则为列优先访问,前$4$行每块与Cache每组一一对应,共有6个连续4行,若采用LRU替换算法,则每次替换主存块时都将最先调入的主存块调出,故每次都不命中,命中率为0。
\section*{x86中的过程调用与返回:}
\noindent
call指令的作用是将相邻的下一条指令地址压栈(push eip),并根据给出的寻址方式无条件跳转到相应指令地址,执行该指令;\\
ret指令的作用是弹出栈顶保存的指令地址,并无条件转移到该地址,执行该指令。\\
x86过程调用与返回的完整过程(通常情况ABI):\\
调用者:
\begin{itemize}
\item 保存调用者保存寄存器中的值到主存栈中并将参数压入传参寄存器或主存栈中;
\item 执行call指令(返回地址压栈,跳转到被调用过程入口地址);
\item 调用经ret返回后,恢复栈中调用者保存寄存器的值到相应寄存器并出栈调用参数。
\end{itemize}
被调用者:
\begin{itemize}
\item 将栈基指针压栈,然后置栈基指针为栈顶指针的值;
\item 若需要使用到被调用者保存寄存器则需要将它们的值压栈;
\item 被调用进程主要部分;
\item 将返回值存入相应寄存器或压栈;
\item 将被调用者保存寄存器值出栈;
\item 置栈顶指针为栈基指针的值,出栈并将出栈的值置为栈基指针 (leave指令);
\item 执行ret指令(返回地址出栈,跳转到该地址)。
\end{itemize}
被调用者保存寄存器是被调用者需要保存的寄存器,调用者保存寄存器是调用者需要保存的寄存器。
\end{document}
