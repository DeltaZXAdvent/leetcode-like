\documentclass{article}
\begin{document}
\section*{Match the longest word in dictionary not necessarily starting from the beginning}
Solving this problem is due to misunderstanding of the original problem.
It should be equivalent to the Aho--Corasick algorithm.
\small
\begin{verbatim}
Maintain a data structure resembling a trie, with extra edges from nodes for mismatches.
The data structure can be implemented in graph (with a root node, used as an automaton).
The data in the graph or trie:
  Nodes: Depth, the depth of the root being 0
         The length of the last complete match, if none, set 0
         Partial match length (maybe in a seperate location)
  Edges: A char_type value denoting the matched character,
         which can have a value representing mismatch
The trie can be implemented with in-order next pointers.

The algorithm:
Step 1, set up the nodes:
  maintain a current node, initialized with the root node
  for each word in dictionary do
    for each character c in word do
      if the current node has en edge with value c
        follow the edge, update the current node
      else
        allocate a new node with incremented depth and a last match length same as the old node
        create an edge from the old node to the new node with value c,
        update the current node
    set the last match length of the current node to be the depth

(Possibly contains some delicacies wrongly dealt with)
Step 2, calculate the mismatch goto node for every node as in the Knuth--Morris-Pratt algorithm
  set the partial match length of the root node and the nodes of depth 1 to be 0
  for each node from the root node, do an in-order traversal
    calculate the partial match length of the current node according to the one of its parent, and
    update the mismatch edge

Step 3, match against a character stream
  maintain a current node and a current matched (not necessarily completely) text, and
    a last complete match node initialized with NULL
    --this is the current state of the automaton
  fetch the next character ch
    if there is a corresponding edge, advance through it
      if the new current node is a complete match node
        update the last complete match length
    else (mismatch)
      if the last complete match node is not NULL
        move back to that node, update the current matched text, and
        output the match, and put back the additional text
      else
        follow the mismatch edge, update the states (...)

\end{verbatim}
As we can see the above example, an algorithm may often need addtional augmented data to make it quicker.
But you can always have a rough overview on what data is available, and then add it to the original data structure.

For example, the ``put back'' operation of the character stream.
\end{document}